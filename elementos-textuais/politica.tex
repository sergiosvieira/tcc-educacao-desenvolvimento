\chapter{POLÍTICA NACIONAL DE TRANSFORMAÇÃO DIGITAL E INTELIGÊNCIA ARTIFICIAL}
\label{chap:politica}

A Política de Transformação Digital e Inteligência Artificial é uma iniciativa do Ministério da Tecnologia de Informação e Comunicação da Colômbia e foi publicada em 2019 para fomentar a criação de valor socioeconômico através do uso estratégico de tecnologias digitais. O objetivo é estimular o uso dessas inovações tecnológicas nos setores públicos e privados e se adequar aos novos desafios apresentados pela Quarta Revolução Industrial (4RI), além de promover o bem-estar de seus cidadãos \cite{de2019politica}.

Segundo \cite{schwab2019quarta}, o termo ``Indústria 4.0'' foi cunhado em 2011 em uma feira de tecnologia industrial na Alemanha para descrever como esta indústria hiperconectada iria revolucionar a organização das cadeias globais de produção, porém, o termo Revolução Industrial 4.0 denota uma visão mais abrangente como explica Schwab.

\begin{citacao}
A quarta revolução industrial, no entanto, não diz respeito apenas a sistemas e máquinas inteligentes e conectadas. Seu escopo é muito mais amplo. Ondas de novas descobertas ocorrem simultaneamente em áreas que vão desde o sequenciamento genético até a nanotecnologia, das energias renováveis à computação quântica. O que torna a quarta revolução industrial fundamentalmente diferente das anteriores é a fusão dessas tecnologias e a interação entre os domínios físicos, digitais e biológicos. \cite{schwab2019quarta}
\end{citacao}

A busca pela superação dos desafios apresentados pela 4RI, segundo \cite{de2019politica}, são importantíssimos para o bem-estar socioeconômico e isto é evidenciado em \cite{schwab2019quarta}, uma vez que a velocidade e o alcance das inovações tecnológicas são inimagináveis. Empresas como a Apple que lançou o primeiro iPhone em 2007, passa a possuir 1 bilhão de iPhones ativos no mundo em setembro de 2020 \cite{felipejunqueira}.

A transformação digital é um fenômeno que está mudando a sociedade, e isso traz um importante desafio para os países da América Latina. No caso da Colômbia, segundo \cite{de2019politica}, foram identificadas alguns impeditivos que precisavam ser superadas, como: reduzir as barreiras que impedem a adoção das tecnologias digitais nas instituições públicas e privadas, gerar condições propícias para que a inovação digital crie novos processos e produtos tanto na iniciativa pública quanto na privada, fortalecer as competências do capital humano para a 4RI e desenvolver condições que permitam a sociedade estar preparada para grandes mudanças econômicas e sociais que a Inteligência Artificial (IA) traz, reconhecendo-a como fundamental na transformação digital sem ignorar a importância de outras tecnologias digitais 4RI (Internet das Coisas, robótica, computação quântica etc).

A inteligência artificial, em comparação com as outras ciências é bem recente, seu estudo foi iniciado após a 2ª Guerra Mundial e, segundo \cite{gomes2010inteligencia}, abrange vários subcampos do conhecimento, desde aprendizado e percepção de forma geral, até a execução de tarefas específicas como jogos de xadrez, diagnóstico de doenças, identificação de fraudes financeiras etc. Ela foi desenvolvida para automatizar tarefas intelectuais e é bastante relevante em diversas atividades humanas. A IA é utilizada para resolver problemas que geralmente são resolvidos através da inteligência humana.

Segundo \cite{de2019politica}, a IA terá prioridade no desenvolvimento deste planejamento digital, uma das justificativas para isso é que, segundo \cite{andrews2017predicts}, ela será prioridade para mais de 30\% empresas e que as habilidades mais exigidas serão aquelas relacionadas à IA.

Neste contexto, é necessário trabalhar em várias frentes -- políticas sociais de acesso à Internet, conscientização do benefício do uso de tais tecnologias, por exemplo -- para alcançar o domínio destas inovações tecnológicas e para isto são necessárias ações para geração de competências digitais.

A formação do talento humano capaz de lidar com estes desafios fica a cargo de uma educação tecnológica e transversal com foco na melhoria do bem-estar social e que também inclua a população pobre e vulnerável e em áreas rurais. Para que seja viável a busca por uma emancipação tecnológica digital, se faz necessário o estímulo dessas competências tecnológicas em toda a população \cite{de2019politica}.

Dentre as diversas iniciativas apresentadas na Política Nacional para Transformação Digital e Inteligência Artificial, há uma seção que trata especificamente sobre o incentivo necessário para criação de habilidades digitais (IoT - Internet das coisas, IA - Inteligência Artificial, Blockchain etc). Este incentivo passou por diversas fases, incluindo um programa de doação massiva de computadores para colégios públicos \cite{compes3063} -- que foi atualizada em 2017, para que fosse dada maior ênfase a entrega de computadores na região rural--, facilitar o acesso irrestrito dos seus habitantes à uma `'Sociedade do Conhecimento'', que é aquela onde as tecnologias desempenham um papel relevante nas atividades sociais, econômicas e culturais \cite{informationsociety}, e em 2019, segundo \cite{de2019politica}, foram introduzidas mudanças significativas através de princípios orientadores para a políticas de TIC (Tecnologia da Informação e Comunicação), focando o bem-estar social, em especial da população pobre e vulnerável, e das áreas rurais.

Dentre as ações relacionadas à educação apresentadas em \cite{de2019politica} para o de fortalecimento das competências do capital humano em relação às habilidades digitais, destacam-se:

\begin{itemize}
    \item Delineamento das diretrizes curriculares de projetos pedagógicos que promovam as competências necessárias para atender as demandas da 4RI, principalmente a respeito da IA.
   \item Desenvolvimento e promoção de um projeto de uma estratégia que estimule a criação de ambientes de aprendizagem que incentive o aprimoramento de competências tecnológicas, científicas e socioemocionais, nos jovens e em seus familiares, de forma comunitária, para interagir com as tecnologias 4RI emergentes.
    \item Promoção das habilidades e metodologias mais eficazes na adoção da IA nas instituições, além de promover treinamentos em IA para a população em geral.
    \item Criação de um mercado de inteligência artificial através do conhecimento gerado pelas universidades onde serão promovidos projetos acadêmicos desenvolvidos nacionalmente que se relacionam positivamente com as empresas privadas.
    \item Acesso contínuo ao conhecim\cite{cerf1993internet}ento da comunidade internacional através de intercâmbio constante com as principais entidades internacionais em IA para permanente atualização e parceria.
   \item Criação e implantação de um plano piloto para identificar os alunos com maiores habilidades nas disciplinas de Ciências, Tecnologia, Engenharia, Matemática e Artes (STEM+A em inglês) em instituições de ensino do país.
\end{itemize}

O plano de transformação digital apresentado neste trabalho busca transformar a Colômbia em um pólo de desenvolvimento em IA através de diversas ações, por exemplo: criação de um mercado nacional de inovação tecnológica, popularização da IA no setor público e privado, ampliação do acesso à tecnologias digitais, consolidação do apoio das universidades locais, transferência de conhecimento através de parcerias internacionais, delineamento educacional para geração de habilidades digitais e busca por novos talentos. A base de sustentação destas ações é a educação pautada em um plano de desenvolvimento nacional. No entanto, não há como afirmar que esta política obterá resultados positivos, uma vez que ela aposta pesadamente em Inteligência Artificial para mudar a realidade local. O mercado de inovação tecnológica é bastante volátil, a necessidade do mercado por profissionais de I.A pode não ser mais uma tendência em alguns anos, ter uma educação de base forte em conjunto de uma universidade que receba investimentos em pesquisa e inovação em várias especialidades pode ser uma estratégia mais efetiva na busca por um protagonismo tecnológico.




