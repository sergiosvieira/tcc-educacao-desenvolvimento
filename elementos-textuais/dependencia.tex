\chapter{A DEPENDÊNCIA TECNOLÓGICA DOS PAÍSES DA AMÉRICA LATINA}
\label{chap:metodologia}

A dependência do capitalismo estrangeiro na América Latina está bastante ligada a substituição dos produtos importados por produtos nacionais devido à forte desindustrialização de parques diversificados e modernos. Segundo \cite{bambirra2012}, o parque industrial da América Latina foi construído quando políticas de contenção das importações geraram uma demanda por produtos, surgindo daí a oportunidade de crescimento das indústrias nacionais com impulsionamento da economia de mercado.  Além da contenção de importações, fatores como a primeira guerra mundial contribuíram para o aumento de uma demanda não atendida por produtos, o que impulsionou a industrialização da América Latina. O lucro obtido pelas exportações era então utilizado para compra de novo maquinário estrangeiro com alto valor agregado para modernizar as indústrias locais, tornando-se em um ciclo vicioso.

Nesse processo, o estímulo ao desenvolvimento, a produção na América Latina de matérias-primas de menor valor agregado, como ensina Marini, contribuíram para que o eixo da acumulação na economia industrial, sobretudo dos países centrais passe a depender mais do aumento da capacidade produtiva do trabalho do que simplesmente da exploração do trabalhador \cite{traspadini2005ruy}. Porém o resultado dessa premissa da predominância nos países mais desenvolvidos é a acumulação baseada na exploração do trabalhador nos países emergentes.

O Estado foi imprescindível para a etapa neoliberal do capitalismo tanto na sua implementação como na expansão durante os anos 1980 e 1990 com sua intervenção \cite{valencia2009reestruturaccao}. O poder estatal ajuda o capital com modificações trabalhistas, estruturais, regulatórias, dentre outras, conforme a necessidade do mercado.

Nesse contexto, segundo \cite{fernandes1975revoluccao}, a revolução burguesa brasileira ocorre quando o antigo poder oligárquico une forças com a burguesia comercial emergente para influenciar politicamente as estruturas vigentes através do capital. Desta união de interesses comuns surge um pacto para perpetuar formalmente a dominação de classes. Ela encontra no plano político, através do Estado, uma maneira de obter vantagens econômicas, sociais e políticas.

No entanto, apesar de a burguesia acumular tais poderes, não era o seu maior interesse a modernidade, a expansão econômica e o bem social universal, ela estava comprometida com aquilo que lhe trazia vantagens. Se por um lado a abolição e a concretização do trabalho livre apresentavam-se, por outro, a manutenção do poder oligárquico no interior era mantida e aprofundada. Da mesma forma, o assalariado que produzia determinado produto não conseguia ter acesso ao bem produzido.

A burguesia e a oligarquia poderiam entrar em conflito, no entanto, elas apenas discordavam quando seus interesses particulares eram feridos. A primeira vista, elas aparentavam ser oriundas de mundos diferentes, mas eram faces da mesma moeda, pois a burguesia mantinha valores e laços culturais estreitos com a oligarquia na busca de mais poder para as classes dominantes.

A burguesia brasileira, em certo grau, tinha um espírito inovador, mas ela buscava favorecer o crescimento econômico e a modernização empresarial em vez de buscar também uma evolução em outras áreas, o bem-estar social, por exemplo. Até a década de 1930, a evolução do capitalismo competitivo se deu a partir de concepções neocoloniais, o foco era a exportação de material de baixo valor agregado e importação de produtos de alto requinte e exclusivos, vindos da Europa sobretudo da França \cite{bueno2016gastronomia}.

Nossa burguesia experimentou também uma influência modernizadora externa na busca por produtos exclusivos pelo mundo, mas não democrática. Os interesses externos e internos alinhavam-se para servir como mola propulsora de uma pretensa estabilidade política e econômica para atender aos interesses das elites.

A dependência que houve e que ainda há, ocorre, pois a produção capitalista dominada pelos países centrais nunca foi inteiramente compartilhada com os países dependentes da América Latina. A produção é apresentada de uma maneira na qual os interesses são impostos e modificados segundo as necessidades dos dominantes do sistema, como um novo colonialismo aceito passivamente pelos países periféricos \cite{bambirra2012}.

O capitalismo na América Latina está em constante transformação, mas isso nem sempre resulta em melhores condições sociais. Propagou-se que o subdesenvolvimento desses países era consequência de sua formação colonial e que a única forma deles evoluírem era através do alinhamento político e econômico com os países centrais, como ensina Marini \cite{marini2005140},

% \renewenvironment{quote}
%   {\begin{trivlist} \setlength\leftskip{2cm} \setlength\rightskip{0pt}
%    \item\relax}
%   {\end{trivlist}}

\begin{citacao}
 A colônia produtora de metais preciosos e gêneros exóticos, a América Latina, contribuiu em um primeiro momento com o aumento do fluxo de mercadorias e a expansão dos meios de pagamento, que, ao mesmo tempo em que permitiam o desenvolvimento do capital comercial e bancário na Europa, sustentaram o sistema manufatureiro europeu e propiciaram o caminho para a criação da grande indústria. \cite{marini2005140}
\end{citacao}

No entanto, alguns intelectuais iniciaram um movimento para repensar o capitalismo latino-americano. Através dos trabalhos de \cite{fernandes1973capitalismo} e \cite{bambirra2012} foram observadas novas perspectivas nas quais negavam que os países latino-americanos eram subdesenvolvidos por terem sido colonizados e portanto não teriam autonomia para erguer uma nação modernizada e que na verdade eles estavam subordinados a uma dependência capitalista aos centros homogênicos que ditavam quando e como suas economias poderiam desenvolver-se \cite{bambirra2012}.

Os países da América latina encontravam-se, na década de 1950, inebriados pela política internacional que afirmava que para que eles pudessem crescer era necessário aproveitar suas características territoriais e de produção no fornecimento de commodities para os países industrializados. Propagavam a ideia de que cada um aproveitaria suas fortalezas, e através de acordos comerciais, cresceriam igualmente. No entanto, na prática, essa política de dependência mostrou-se vantajosa apenas para os países do capitalismo central. O valor agregado dos produtos industrializados era bastante superior aos produtos exportados pela América Latina.

Neste período, a Comissão Econômica para a América Latina e o Caribe (CEPAL) começou a analisar o relacionamento econômico de outra forma \cite{bresser2005iseb}. O intuito agora era o de incentivar a industrialização na América latina e para isto era necessário então importar maquinário para desenvolver as indústrias locais em vez de importar apenas produtos manufaturados. Esta mudança de visão, tinha por objetivo fazer com que a burguesia tradicional financiasse essa industrialização e fosse catalisadora de uma transformação que pudesse levá-los a um desenvolvimento autônomo e consequentemente a melhorias sociais. Infelizmente essa mudança não se concretizou. A dependência tecnológica dos países da América Latina com os países centrais fazia com que o custo dessa modernização fosse pago através da superexploração dos trabalhadores latinos.

Nos anos seguintes foram desenvolvidas novas teorias dentre as quais se destacaram: a Teoria Marxista da Dependência - que propagava o rompimento da parceria entre os países do capital dependente e os do capital central para mudar o modelo de nacionalista para socialista - e a Teoria Positiva da Dependência que preconizava que era vantajosa a associação entre os países dependentes e centrais uma vez que eles não haviam desenvolvido um processo tecnológico capaz de industrializar-los rapidamente, assim, era mais vantajoso se associar com a burguesia industrial do países centrais para mais uma vez usufruir de suas tecnologias e com isto, ganhar-se-ia mais atraindo investidores internacionais que investindo em tecnologias nacionais.

Em meados da década de 1970 o neoliberalismo econômico desponta como uma alternativa às ideias anteriores. Acreditava-se que haveria um ganho na sociedade através do desenvolvimento das relações mercantis. O aumento da competitividade e da produtividade eram valores que buscavam melhorar a qualidade dos produtos nacionais ao mesmo tempo que fomentava uma melhor distribuição de renda. No entanto, o neoliberalismo na América latina acaba agravando a dependência tecnológica e a transferência de valor para os países centrais. Novamente a conta é paga pelos trabalhadores.

Um dos principais aspectos da consolidação do poder dos países centrais sobre a América Latina é sua capacidade de produzir bens tecnológicos relevantes. Esta vantagem acaba por consolidar sua influência na economia, na política e até mesmo no poderio militar. Segundo \cite{quijano2005colonialidade}, essa vantagem surge através da expansão de ideários coloniais de poder e de saber onde o eurocentrismo se destaca.

Os eurocentristas, como exibido no trabalho de \cite{quijano2005colonialidade}, justificaram a subjugação dos povos nativos da América Latina afirmando que o modelo de civilização a ser buscado era o europeu, e assim era natural que o homem caucasiano teria por direito inato o dever de guiar as outras culturas ditas menos desenvolvidas segundo seu modelo de civilização. Este ideário pode ser visto ainda hoje no sistema capitalista moderno, onde diversos setores de nossa sociedade, incluindo as Universidades, buscam copiar esse modelo de ``sucesso'' sem considerar as particularidades dos povos latino-americanos.

Nesse contexto, a teoria deocolonial aponta que além da visão eurocentrista, onde já existe esse direito inato de subjugação, também existe a colonialidade que contribui para fortalecer esta sujeição dos povos da América Latina.

A colonialidade diz respeito a forma na qual as instituições e a sociedade da América Latina absorvem a ideia de que o saber bom é oriundo da Europa. O modelo de dominação dos países centrais não estaria completo se fosse apenas político e econômico. Há ainda a necessidade de impor o modo de vida ocidental como o único capaz de guiar as sociedades menos desenvolvidas.

No contexto atual, a América Latina continua dependente de suas exportações e dependente de produtos com maior valor agregado dos países tecnologicamente avançados, haja vista a diferença na balança comercial entre os países latinos e os países mais desenvolvidos. Uma característica bem relevante nesta relação comercial é que os países europeus realizam uma exportação maior em bens manufaturados, quanto que os países sul-americanos uma exportação de maior intensidade em bens primários, conforme mostra \cite{serra2013}.

