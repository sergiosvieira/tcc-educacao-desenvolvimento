\chapter{Introdução}
\label{cap:introducao}
%\onehalfspacing

É importantíssimo o papel que os países centrais dão à educação no desenvolvimento científico-tecnológico \cite{lima2009politica}. Investir em educação não deve ser encarado apenas como uma devolutiva do estado, e sim como uma estratégia para estimular o desenvolvimento socioeconômico de uma nação. Se bem planejado, o investimento em educação de base e universitário, atrelado a um desenvolvimento das forças produtivas através da Ciência e Tecnologia e uma mudança profunda na relação socioeconômica entre as classes dominantes e as populares, poderia contribuir para mitigar a dependência tecnológica dos países da América Latina com os países centrais \cite{lima2009politica}.

O uso de tecnologias digitais é reconhecidamente um dos pilares de desenvolvimento socioeconômicos atuais \cite{aragon2016notas} e pode contribuir para que uma nação não desenvolvida tecnologicamente busque um protagonismo no cenário internacional \cite{jardim2018}. Neste sentido, é preciso alinhar os objetivos de um plano de desenvolvimento nacional com base na educação, justiça socioeconômica, aprimoramento dos meios de produção com base na ciência e tecnologia e defesa dos interesses nacionais frente aos interesses externos. Desta forma, o objetivo deste trabalho é o de analisar a importância da educação na busca por emancipação tecnológica na América Latina, em especial o plano de desenvolvimento nacional pretendido pela Colômbia intitulado ``Política Nacional para la Transformación Digital e Inteligencia Artificial’’ \cite{de2019politica}. No caso colombiano, a desvalorização dos preços internacionais das commodities, fez com que ela buscasse novas estratégias de crescimento. O caminho escolhido foi o da transformação digital que busca transformá-la de uma economia baseada em commodities e recursos naturais em uma baseada no conhecimento e na agregação de valor dos serviços tecnológicos.

Inicialmente discorre-se sobre como se deu, do ponto de vista histórico, a consolidação da dependência econômica/tecnológica dos países da América Latina. Em seguida, apresenta-se qual o papel da Pesquisa e Desenvolvimento no protagonismo dos países desenvolvidos e qual a importância que eles dão à educação. Por último, analisa-se como o Plano de Transformação Digital e Inteligência Artificial busca mitigar os problemas relacionados à dependência e quais as diretrizes relacionadas com a educação nesta busca por uma transformação social, econômica e cultural.

