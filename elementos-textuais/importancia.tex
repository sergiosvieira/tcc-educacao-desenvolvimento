\chapter{A IMPORTÂNCIA DA EDUCAÇÃO NO DESENVOLVIMENTO CIENTÍFICO TECNOLÓGICO}
\label{chap:importancia}

A tecnologia, no sentido de conjunto de saberes, vem ao longo da história humana sendo utilizada para desenvolver a sociedade como um todo. O desenvolvimento humano pode ser observado através destas mudanças históricas ocasionadas pela aplicação da tecnologia nos meios de produção. O sistema capitalista, em sua essência, é pautado pela acumulação da técnica e métodos para produção de excedente, diferente da força de trabalho, que busca atender inicialmente às suas necessidades básicas. Neste sentido, a exposição de \cite{vieira2005conceito} diz:

\begin{citacao}
 Deste modo, a tecnologia e os maquinismos de qualquer espécie que a materializam assumem duplo e profundo significado: o de ser ao mesmo tempo manifestação da razão humana no grau de desenvolvimento que alcançam em cada momento e em certa comunidade nacional sob forma de consciência para si: e o de fornecer o condicionamento objetivo, a força o ``motor'' do processo de evolução da mesma razão.
\end{citacao}

A revolução técnico-científica pressupõe a busca do homem em utilizar desta tecnologia em sua libertação do trabalho braçal para que ele busque a sua satisfação além das suas necessidades básicas. No entanto, quando o capital privado busca a acumulação, a força de trabalho que necessita adquirir tais produtos acaba se tornando mais dependente da venda da sua capacidade produtiva, e que acaba lhe privando de usufruir outras necessidades humanas que não estão relacionadas com suas necessidades de subsistência.

Segundo \cite{vieira2005conceito}, uma das formas de conseguir justiça social é através da popularização da tecnologia na vida do homem. Mas para que isso ocorra plenamente, a ciência não pode ser considerada um mero produto pertencente das elites burguesas. A ciência deve ser utilizada como meio de humanização. Ela precisa ser gratuita e oferecida de forma ampla sem beneficiar apenas grupos historicamente privilegiados. O saber utilizado no desenvolvimento tecnológico é milenar e assim não há justiça quando grupos se utilizam desse saber para capitanear as evoluções técnico-científicas.

A tecnologia pode ser considerada a materialização da ciência quando aplicada ao processo produtivo de forma geral. Ela não apenas constrói máquinas e instrumentos e é através dela que as atividades humanas, em diversas esferas da sociedade, podem ser organizadas, planejadas e construídas. Sua evolução não é resultado apenas do gênio individual e sim de um esforço conjunto na acumulação de conhecimentos conforme sugere Álvaro Pinto.

\begin{citacao}
O ato de descoberta de algum conhecimento científico ou de intervenção de algum maquinismo novo, por mais que, ao exame imediato e local, pareça nascer exclusivamente do gênio individual, do espírito iluminado, quando visto na sequência do movimento expansivo da tecnologia desde os alvores, indica pertencer ao mesmo processo biológico de desenvolvimento da razão, por força dos condicionamentos sociais a que agora aludimos. Depende da acumulação dos conhecimentos e das técnicas anteriores, evidentemente com as características do determinismo não mecânico, não linearmente causal, mas dialético, de acordo com o qual a razão revela-se simultaneamente um resultado em contínua produção e uma mediação de ordem superior nesse mesmo processo. \cite{vieira2005conceito}
\end{citacao}

A inovação tecnológica possibilita a realização de feitos inimagináveis. No entanto, ela pode consolidar ainda mais o controle que os países centrais possuem sobre os países da América Latina. Segundo \cite{freeman1975teoria}, a inovação tecnológica pode desenvolver estruturas de bens e serviços em escala mundial, e isto necessariamente não resultaria em melhorias na qualidade de vida em países, centrais ou não, uma vez que cada um possui características diferentes, demandando respostas diferentes para solucionar seus problemas \cite{leme2019maquina}.

No contexto de desenvolvimento tecnológico, existe uma questão importante apresentada em \cite{faoro1992questao} que trata sobre a diferença entre modernidade e modernização. Tal entendimento contribui para compreendermos melhor o papel da educação no desenvolvimento tecnológico tanto no Brasil quanto em outros países da América Latina.

\begin{citacao}
 O Brasil padece de ímpetos de modernização, através dos quais se tenta queimar etapas no processo de desenvolvimento. Uma nova modernização sepulta a anterior e nenhuma consegue fazer com que o país encontre o caminho para o desenvolvimento. Impostas por elites pseudo dissidentes em favor dos seus interesses, essas modernizações mantêm a maioria da população alijada de benefícios sociais elementares. \cite{faoro1992questao}
\end{citacao}

Segundo \cite{faoro1992questao}, ``A modernização é um traço de linhas duplas: a linha do paradigma e o risco do país modernizável. Quando ela, a modernização, se instaura, como ação voluntária, quem a dirige é um grupo ou classe dirigente''. Portanto, nem sempre ao pregarmos a modernização nos países da América Latina, iremos contribuir para seu desenvolvimento tecnológico, uma vez que a modernização pode ser dirigida por uma elite dominante que pode não contribuir no desenvolvimento de soluções para problemas comunitários.

Ainda segundo \cite{faoro1992questao}, ``Na modernidade, a elite, o estamento, as classes, dizemos para simplificar, as classes dirigentes, coordenam e organizam um movimento. Não o dirigem, conduzem ou promovem, como na modernização.'' e segundo \cite{lima2009politica}, ``Na modernidade, a sociedade se transforma, atualiza, aperfeiçoa, desenvolve; na modernização, que copia os modelos sem alterar o sistema de poder, a sociedade acumula, soma, progride, mas não se desenvolve''.

Em um contexto econômico, as ideias apresentadas em \cite{dupas2001economia} corroboram com a diferenciação apresentada em \cite{faoro1992questao}. Segundo \cite{dupas2001economia}, as condições de produção globais, acirradas pela competição do capital, demandam uma abordagem mais modernizada que modernizadora nos pólos industriais dos países centrais e isto contribuiu para um nova reestruturação de economias nacionais com o objetivo de aprimorar a forma de produzir. Este aprimoramento se daria através de uma articulação entre ciência e tecnologia. Esta nova abordagem de produção impulsionou a criação de departamentos de Pesquisa e Desenvolvimento (P\&D) nas indústrias dos países centrais e consequentemente aumentou a busca por cientistas especializados para desenvolver estas inovações tecnológicas.

Segundo \cite{lima2009politica}, o desenvolvimento científico-tecnológico é fundamental para o crescimento econômico nos países centrais e neles são criadas políticas públicas que financiam pesquisas onde são envolvidas as universidades e a iniciativa privada. A educação neste sentido, torna-se necessária pois a partir dela, as demandas por cientistas, pelos departamentos de P\&D, poderiam ser atendidas.

Segundo as ideias desenvolvidas em \cite{faoro1992questao} e \cite{lima2009politica}, o desenvolvimento tecnológico pode ser conseguido através da modernidade que está diretamente relacionada ao desenvolvimento social, econômico e intelectual de uma nação \cite{lima2009politica}. Nos últimos 5 anos, segundo o relatório da American Association for the Advancement of Science \cite{american2021}, a China aumentou, em média, o investimento público em P\&D em 9\%, a Alemanha em 6,6\% e a União Europeia em 5,8\%, um pouco acima do crescimento de 5,5\% dos EUA. Isto posto, é notória a importância que os países desenvolvidos dão a Pesquisa e Desenvolvimento para modernização de seus modos de produção em busca de consolidar ainda mais o papel de protagonismo global.

